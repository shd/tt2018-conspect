\documentclass[12pt, a4paper]{article}
\usepackage[T1,T2A]{fontenc}
\usepackage[utf8]{inputenc}
\usepackage[english,russian]{babel}
\usepackage{amsmath}
\usepackage{amsfonts}
\usepackage{amssymb}
\usepackage{makeidx}
\begin{document}
	\begin{titlepage}
		\title{Лекция 5 \\ Изоморфизм Карри-Ховарда (завершение), Унификация}
		\date{}
	\end{titlepage}
		\maketitle
	\par \textbf{Определение}
	\\	
	Изоморфизм Карри-Ховарда
	\begin{enumerate}
		\item $\Gamma$ $\vdash$ M:$\sigma$ влечет |$\Gamma$|$\vdash$ $\sigma$
		\item $\Gamma$ $\vdash$ $\sigma$, то существует M и существует $\Delta$, такое что |$\Delta$|=$\Gamma$, что $\Delta$ $\vdash$M:$\sigma$, где $\Delta$=\{$x_{\sigma}$:$\sigma$|$\sigma$ $\in$ $\Gamma$  \}
	\end{enumerate}
	Рассмотрим пример:
	\{f:$\alpha\rightarrow\beta$, x:$\beta$\}$\vdash$fx:$\beta$ \\Применив изоморфизм Карри-Ховарда получим: \{$\alpha\rightarrow\beta$, $\beta$\}$\vdash\beta$\\
\par П.1 доказывается индукцией по длине выражения т.е. есть 3 правила вывода. убирая P и Q.
\par П.2 доказывается аналогичным способом но действия обратные.\\
Т.е. отношения между типами в системе типов могут рассматриваться как образ отношений между высказываниями в логической системе, и наоборот.
\\
\\
\par \textbf{Определение}
\par расширенный полином определяется формулой:
	\[
    E(p,q)= 
		\begin{cases}
    C,& \text{if }p=q=0\\
    p_1(p),& \text{if }q=0\\
    p_2(q),& \text{if }p=0\\
    p_3(p,q),& \text{if } p,q\neq0
		\end{cases}
	\]
	\[\text{, где }C-\text{константа, }\\p_1,p_2,p_3-\text{выражения, составленные из }*,+,p,q\text{ и констант}\]
	по сути расширенный полином это множество функций над натуральными числами(черчевскими нумералами)
	\par \textbf{Теорема}
	\par Функция определенная в просто типизиреумом $\lambda$ исчислении соответствует расширенному полиному.
\par \textbf{Ссылки}
\begin{enumerate}
\item https://www.quora.com/What-is-an-intuitive-explanation-of-the-Curry-Howard-correspondence
\item https://habr.com/post/269907/
\item https://arxiv.org/pdf/cs/0701022.pdf
\end{enumerate}

\end{document}