\section{Лекция 4}

\subsection{Расширение просто типизированного $\lambda$-исчисления до изоморфного ИИВ}

Заметим, что между просто типизированным по Карри $\lambda$-исчислением и имликационным фрагментом ИИВ существует изоморфизм, но при этом в просто типизированном $\lambda$-исчислении нет аналогов лжи, а также связок $\vee$ и $\&$.

Для установления полного изоморфизма между ИИВ и просто типизированным $\lambda$-исчислением введём три необходимые для установления этого изоморфизма сущности:

\begin{enumerate}
	\item Тип "Ложь" ($\bot$)
	
	\item Тип упорядоченной пары $A\&B$, соответсвующий	логическому "И"
	
	\item Алгебраический тип $A | B$, соттветсвующий логическому "ИЛИ"
\end{enumerate}

\subparagraph{Тип $\bot$}

Введём тип $\bot$, соттветствующий лжи в ИИВ. Поскольку из лжи может следовать что угодно, добавим в исчисление новое правило вывода

\[
\infer{\Gamma \vdash A : \tau}{\Gamma \vdash A: \bot}
\]

То есть выражение, типизированное как $\tau$, может быть типизированно так же любым другим типом.

В программировании аналогом этого типа может являться тип \mintinline{scala}{Nothing}, который является подтипом любого другого типа.