\documentclass[12pt,a4paper,oneside]{article}
\usepackage[english,russian]{babel}
\usepackage[utf8]{inputenc}
\usepackage{amsmath}
\usepackage{amsthm}
\usepackage{bnf}
\usepackage{amssymb}
\usepackage{enumerate}
\usepackage{mathtext}
\usepackage{mathtools}
\usepackage{mathabx}
\usepackage[left=2cm,right=2cm,top=2cm,bottom=2cm,bindingoffset=0cm]{geometry}
\usepackage{proof}



\title{Введение в Теорию Типов\\\it{Конспект лекций}}
\author{Штукенберг Д.~Г.\\Университет ИТМО}
\DeclareMathOperator{\FV}{FV}

\begin{document}

\theoremstyle{definition}
\newtheorem{definition}{Определение}[section]
\newtheorem{cons}{Следствие}[section]
\newtheorem*{example}{Пример}
\newtheorem{theorem}{Теорема}[section]
\newtheorem{axiom}{Аксиома}[section]
\newtheorem{lemma}{Лемма}[section]
\newtheorem{statement}{Утверждение}[section]

\newcommand{\comb}[1]{\operatorname{\mathcal{#1}}}
\newcommand{\func}[1]{\operatorname{#1}}
\newcommand{\reduction}[1]{{\color{OrangeRed}#1}}
\newcommand{\set}[1]{\left\{#1\right\}}

\def\from#1{\par \parbox{0.7\textwidth}{\par \hfill\raggedleft \it #1}} 

\newenvironment{epigraph}% 
{\begin{list}{}{\setlength{\leftmargin}{0.3\textwidth}}\item[]}% 
{\end{list}} 

\maketitle
\section{Введение}

Данный конспект содержит изложение материалов лекций, рассказанных студентам групп М3336--М3339 
в 2018 году в Университете ИТМО, на Кафедре компьютерных технологий Факультета информационных технологий и прогаммирования.

Конспект подготовили студенты Кафедры: Галкин Егор (лекции 1 и 2),
Илья Кокорин (лекции 3 и 4), Никита Дугинец (лекции 5 и 6), Степан Прудников (лекции 7 и 8).

(возможно, история сложнее)
\section{Лекция 1}

\subsection{$\lambda$-исчисление}

\begin{definition}[$\lambda$-выражение]
	$\lambda$-выражение "--- выражение, удовлетворяющее грамматике:
	\vspace{1mm}
	\begin{bnf}
	\begin{alignat*}{3}
		\Phi &::=& x \\
		| & \left(\Phi\right) \\
		| & \lambda{}x.\Phi \qquad && () \\
		| & \Phi \ \Phi         && () \\
	\end{alignat*}
	\end{bnf}
	\begin{enumerate}
		\item Аппликация левоассациативна.
		\item Абстракции жадные, едят все что могут.
		\vspace{1mm}
		\begin{example}
			$(\lambda{}x.(\lambda{}f.((f x) (f x) \lambda{}y.(y f))))$
		\end{example}
	\end{enumerate}
\end{definition}

Есть понятия связанного и свободного вхождения переменной (аналогично исчислению предикатов).
$\lambda{}x.A$ связывает все свободные вхождения $x$ в $A$.
\begin{definition} 
	Функция $\FV(A)$ "--- множество свободных переменных, входящих в $A$:
	\[
	\FV(A) =
	\begin{cases}
	\set{x}                  & \text{если } A \equiv x \\
	\FV(P) \cup \FV(Q)       & \text{если } A \equiv PQ \\
	\FV(P) \setminus \set{x} & \text{если } A \equiv \lambda x . P
	\end{cases}
	\]
\end{definition}

Договоримся, что:
\begin{itemize}
	\item Переменные "--- $x$, $a$, $b$, $c$.
	\item Термы (части $\lambda$-выражения) "--- $X$, $A$, $B$, $C$.
	\item Фиксированные переменные обозначаются буквами из начала алфавита, метапеременные "--- из конца.
\end{itemize}

На самом деле, смысл в этом есть, $\lambda$-выражение можно понимать как функцию.
Абстракция "--- это функция с аргументом, аппликация "--- это передача аргумента.

\begin{definition}[$\alpha$-эквивалентность]
	$A=_{\alpha}B$, если имеет место одно из следующих условий:
	\begin{enumerate}
		\item $A\equiv{}x$, $B\equiv{}y$ (x,y "--- переменные) и $x\equiv{}y$
		\item $A\equiv{}P_{1}Q_{1}$, $B\equiv{}P_{2}Q_{2}$ и $P_{1}=_{\alpha}P_{2}$, $Q_{1}=_{\alpha}Q_{2}$
		\item $A\lambda{}x.P_{1}$, $B\lambda{}y.P_{2}$и $P_{1} [x\coloneqq{}t] =_{\alpha}P_2 [y\coloneqq{}t]$, где $t$ "--- новая переменная.
	\end{enumerate} 
\end{definition}

\begin{example}
	$\lambda{}x.\lambda{}y.xy=_{\alpha}\lambda{}y.\lambda{}x.yx$.
	\begin{proof} Согласно второму правилу следующие утверждения верны:
		\begin{alignat*}{2}
		\lambda{}y.ty=_{\alpha}\lambda{}x.tx &\implies \lambda{}x.\lambda{}y.xy=_{\alpha}\lambda{}y.\lambda{}x.yx \\
		tz=_{\alpha}tz &\implies \lambda{}y.ty=_{\alpha}\lambda{}x.tx
		\end{alignat*}%
		$tz=_{\alpha}tz$ верно по третьему условию.
	\end{proof}
\end{example}

\begin{definition}[$\beta$-редекс]
	$\beta$-редекс "--- выражение вида: $\left(\lambda{}x.A\right)B$
\end{definition}

\begin{definition}[$\beta$-редукция]
	$A\to_{\beta}B$, если имеет мето одно из следующих условий:
	\begin{enumerate}
		\item $A\equiv{}P_{1}Q_{1}$, $B\equiv{}P_{2}Q_{2}$ и либо $P_{1}=_{\alpha}P_{2}$, $Q_{1}\to_{\beta}Q_{2}$, либо
		$P_{1}\to_{\beta}P_{2}$, $Q_{1}=_{\alpha}Q_{2}$
		\item $A\equiv\left(\lambda{}x.P\right) Q$, $B\equiv P[x\coloneqq{}Q]$ "--- Q свободна для подстановки вместо x в P 
	\end{enumerate}
	\begin{example} 
		$X\to_{\beta}X$, $\left(\lambda{}x.x\right) y\to_{\beta} y$
	\end{example}
	\begin{example}
		 $a \left(\lambda{}x.x\right) y\to_{\beta} a y$
	\end{example}
	\begin{example}
		$A\equiv\lambda{}x.P$, $B\equiv\lambda{}x.Q$, $P\to_{\beta}Q$
	\end{example}
\end{definition}

\subsection{Представление некоторых функций в лямбда исчислении}
Boolean значения легко представить в терминах $\lambda$-исчисления, к примеру
\begin{itemize}
	\item $True$   = $\lambda{}a\lambda{}b.a$ 
	\item $False$  = $\lambda{}a\lambda{}b.b$
\end{itemize}

\newcommand{\If}{\lambda{}c.\lambda{}t.\lambda{}e.(c t) e}
\newcommand{\T}{\lambda{}a\lambda{}b.a}
\newcommand{\F}{\lambda{}a\lambda{}b.b}



Также мы можем выражать и более сложные функции \\

\begin{example}
	$If \ T \ a \ b \to_{\beta} \ a$
	\begin{proof}
		\begin{alignat*}{2}
		 ((\If) \ \T)\ a \ b \to_{\beta} (\lambda{}t.\lambda{}.e(\T) \ t \ e) \ a \ b \to_{\beta} \ \\ (\lambda{}t.\lambda{}.e(\lambda{}b.t) \ e) \ a \ b \to_{\beta} \ (\lambda{}t.\lambda{}e.t) \ a \ b \to_{\beta} \ (\lambda{}e.a) \ b \to_{\beta} \ a
		\end{alignat*}
	\end{proof}
\end{example}

Как мы видим If true действительно возвращает результат первой ветки.


Другие логические операции:
\[
	Not = \lambda{}a.a \ F \ T \qquad
	And = \lambda{}a.\lambda{}b.a \ b \ F \qquad
	Or = \lambda{}a.\lambda{}b.a \ T \ b \qquad
\]



\subsection{Черчевские нумералы}

\begin{definition}[черчевский нумерал]
	\[
		\overline{n} = \lambda{}f.\lambda{}x.f^{n} x \text{, \quad где\quad}
		f^{n} x = 
		\begin{cases}
			f\left(f^{n-1} x\right) & \text{при } n > 0 \\
			x 						& \text{при } n = 0
		\end{cases}
	\]
\end{definition}





\section{Лекция 2}

\subsection{Формализация $\lambda$-термов, классы $\alpha$-эквивалентности термов}

\begin{definition}[$\lambda$-терм]
	Рассмотрим классы эквивалентности $[A]={\alpha}$ \\
	Будем говорить, что $[A]\to_{\beta}[B]$, если $\exists A^{'}\in [A], B^{'} \in [B]$, что $A^{'}\to_{\beta}B^{'}$.
\end{definition}

\begin{lemma}
	$=_{\alpha}$ "--- отношение эквивалентности.
\end{lemma}

Пусть в A есть $\beta$-редекc $\lambda{}x.Q$, но $P[x\coloneqq{}Q]$ не может быть,
тогда найдем $y\notin V[P]$, $y\notin V[Q]$. Сделаем замену $P[x\coloneqq{}y]$.
Тогда замена $P[x\coloneqq{}y][y\coloneqq{}Q]$ допустима.

\begin{lemma}
	$P[x\coloneqq{}y]=_{\alpha}P[x\coloneqq{}y][y\coloneqq{}Q]$, если замена допустима.
\end{lemma}

\subsection{Нормальная форма, $\lambda$-выражения без нормальной формы, комбинаторы $K$, $I$, $\Omega$}

\begin{definition}
	Нормальня форма "--- это $\lambda$-выражение без $\beta$-редексов.
\end{definition}

\begin{lemma}
	$\lambda$-выражение $A$ в нормальноф форме, т.и.т.т, когда $\nexists{}B$, что $A\to_{\beta}B$.
\end{lemma}

\begin{definition}
	$A$ "--- Н.Ф $B$, если $\exists A_{1}...A_{n}$, что $B=_{\alpha}A_{1}\to_{\beta}A_{2}\to_{\beta}...\to_{\beta}A_{n}=_{\alpha}A$.
\end{definition}

\begin{definition}
	Комбинатор "--- $\lambda$-выражение без свободных переменных.
\end{definition}

\begin{definition} 
	\hfill
	\begin{itemize}
		\item $I=\lambda{}x.x$ (Identitant)
		\item $K=\lambda{}a.\lambda{}b.a$ (Konstanz)
		\item $\Omega = (\lambda{}x.xx)(\lambda{}x.xx)$
	\end{itemize}
\end{definition}

\begin{lemma}
	$\Omega$ "--- не имеет нормальной формы.
\end{lemma}

\begin{proof}
	$\Omega\to_{\beta}\Omega$
\end{proof}

\subsection{$\beta$-редуцируемость}

\begin{definition}
	Будем говорить, что $A\twoheadrightarrow_{\beta}B$, если $\exists$ такие $X_{1}..X{n}$, что $A=_{\alpha}X_{1}\to_{\beta}X_{2}\to_{\beta}...\to_{\beta}X_{n-1}\to_{\beta}X_{n}=_{\alpha}B$.
\end{definition}

$\twoheadrightarrow_{\beta}$ "--- рефлексивное и транзитивное замыкание $\to_{\beta}$. $\twoheadrightarrow_{\beta}$ не обязательно приводит к нормальной форме
\begin{example}
	$\Omega\twoheadrightarrow_{\beta}\Omega$
\end{example}

\subsection{Ромбовидное свойство}

\begin{definition}[Ромбовидное свойство]
	Отношение $R$ обладает ромбовидным свойством, если $\forall a,b,c$, таких, что $aRb$, $aRc$, $b\neq{}c$, $\exists{}d$, что $bRd$ и $cRd$. Далее будем обозначать ромбовидное свойство как $<>$.
\end{definition}

\begin{example}
	$(\leq)$ на множестве натуральных чисел обладает $<>$
	$(>)$ не обладает $<>$ на множестве натуральных чисел 
\end{example}

\subsection{Теорема Чёрча-Россер, следствие о единственности нормальной формы}

\begin{theorem}[Черча-Россер]
	$(\twoheadrightarrow_{\beta})$ обладает ромбовидным свойством.
\end{theorem}


\begin{cons}
	Если у $A$ есть Н.Ф, то она единтсвенная с точностью до $(=_{\alpha})$ (переименования переменных).
\end{cons}

\begin{proof}
	Пусть $A\twoheadrightarrow_{\beta}B$ и $A\twoheadrightarrow_{\beta}C$. $B$, $C$ "--- нормальные формы и $B\neq_{\alpha}C$. 
	Тогда по теореме Черча-Россера $\exists{}D$: $B\twoheadrightarrow_{\beta}D$ и $C\twoheadrightarrow_{\beta}D$. Тогда $B=_{\alpha}D$ и $C=_{\alpha} D$ $\Rightarrow$ $B=_{\alpha}C$. Противоречие.
\end{proof}

\begin{lemma}
	Если $B$ "--- Н.Ф, то $\nexists{}Q$: $B\to_{\beta}Q$. Значит если $B\twoheadrightarrow_{\beta}Q$, то количество шагов редукции равно 0.
\end{lemma}

\begin{lemma}
	Если $R$ "--- обладает $<>$, то и $R^{*}$ (транзитивное, рефлексивное замыкание $R$) обладает $R^{*}$.
\end{lemma}

\begin{proof}
	content...
\end{proof}

\begin{lemma}[Грустная лемма]
	$(\to_{\beta})$ не обладает $<>$
\end{lemma}

\begin{definition}[Параллельная $\beta$-редукция]
	$A\rightrightarrows_{\beta}B$, если
	\begin{enumerate}
		\item $A=_{\alpha}B$
		\item $A\equiv{}P_{1}Q_{1}$, $B\equiv{}P_{2}Q_{2}$ и $P_{1}\rightrightarrows_{\beta}P_{2}$, $Q_{1}\rightrightarrows_{\beta}Q_{2}$
		\item $A\equiv{}\lambda{}x.P_{1}$, $B\equiv{}\lambda{}x.P_{2}$ и 
		$P_{1}\rightrightarrows_{\beta}P_{2}$
		\item $A=_{\alpha}(\lambda{}x.P)Q$, $B=_{\alpha}P[x\coloneqq{}Q]$
	\end{enumerate}
\end{definition}

\begin{lemma}
	$(\rightrightarrows_{\beta})$ обладает $<>$
\end{lemma}

$P_{1}\rightrightarrows_{\beta}P_{2}$ и $Q_{1}\rightrightarrows_{\beta}Q_{2}$, то $P_{1}[x\coloneqq{}Q_{1}]\rightrightarrows_{\beta}P_{2}[x\coloneqq{}Q2]$

\subsection{Нормальный и аппликативный порядок вычислений}

\begin{example}
	Выражение $KI\Omega$ можно редуцировать двумя способами:
	\begin{enumerate}
		\item $\comb K \comb I \Omega =_{\alpha} ((\lambda{}a.\lambda{}b.a) I) \Omega \to_{\beta} (\lambda{}b.\comb I)\Omega  \to_{\beta} \comb I$
		\item  $\comb K \comb I \Omega =_{\alpha} ((\lambda{}a.\lambda{}b.a) I)((\lambda{}x.x \ x) (\lambda{}x.x \ x)) \twoheadrightarrow_{\beta} ((\lambda{}a.\lambda{}b.a) I)((\lambda{}x.x \ x) (\lambda{}x.x \ x)) \to_{\beta} \comb K \comb I \Omega $
	\end{enumerate}
	
\end{example}

Как мы видим, в первом случае мы достигли нормальной формы, в то время как во втором мы получаем бесонечную редукцию. Разница двух этих способов в порядке редукции. Первый называется нормальный порядок, а второй аппликативный. 

\begin{definition}[нормальный порядок редукции]
	Редукция самого левого $\beta$-редекса.
\end{definition}

\begin{definition}[аппликативный порядок редукции]
	Редукция самого левого $\beta$-редекса из самых вложенных.
\end{definition}

\begin{statement}
	Если нормальная форма существует, она может быть достигнута нормальным порядком редукции.
\end{statement}
\end{document}
