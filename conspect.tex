\documentclass[12pt, a4paper]{article}
\usepackage{graphicx}
\usepackage{stmaryrd}
\usepackage[T1,T2A]{fontenc}
\usepackage[utf8]{inputenc}
\usepackage[english,russian]{babel}
\usepackage{amsmath}
\usepackage{amsfonts}
\usepackage{amssymb}
\usepackage{makeidx}
\usepackage{verbatim}
\usepackage{tikz}
\usepackage{amsthm}
\usepackage{enumerate}
\usepackage[left=2cm,right=2cm,top=2cm,bottom=2cm,bindingoffset=0cm]{geometry}
\usepackage{proof}
\usetikzlibrary{graphs}
\usetikzlibrary{graphs.standard}
\usetikzlibrary{automata,positioning}
\usepackage{float}
\title{Введение в Теорию Типов\\\it{Конспект лекций}}
\author{Штукенберг Д.~Г.\\Университет ИТМО}

\begin{document}

\theoremstyle{definition}
\newtheorem{definition}{Определение}[section]
\newtheorem{note}{Замечание}[section]
\newtheorem*{example}{Пример}
\newtheorem{theorem}{Теорема}[section]
\newtheorem{axiom}{Аксиома}[section]
\newtheorem{lemma}[theorem]{Лемма}
\newtheorem{statement}{Утверждение}[section]
\newtheorem{oun_paragraph}{Пункт}[section]
\def\from#1{\par \parbox{0.7\textwidth}{\par \hfill\raggedleft \it #1}} 


\newenvironment{epigraph}% 
{\begin{list}{}{\setlength{\leftmargin}{0.3\textwidth}}\item[]}% 
{\end{list}} 

\maketitle

\section{Введение}

Данный конспект содержит изложение материалов лекций, рассказанных студентам групп М3336--М3339 
в 2018 году в Университете ИТМО, на Кафедре компьютерных технологий Факультета информационных технологий и прогаммирования.

Конспект подготовили студенты Кафедры: Галкин Егор (лекции 1 и 2),
Илья Кокорин (лекции 3 и 4), Никита Дугинец (лекции 5 и 6), Степан Прудников (лекции 7 и 8).

(возможно, история сложнее)
\input{lection7}
\input{lection8}

\end{document}
