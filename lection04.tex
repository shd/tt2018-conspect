\section{Лекция 4}

\subsection{Расширение просто типизированного $\lambda$-исчисления \\до изоморфного ИИВ}

Заметим, что между просто типизированным по Карри $\lambda$-исчислением и импликационным фрагментом ИИВ существует изоморфизм, но при этом в просто типизированном $\lambda$-исчислении нет аналогов лжи, а также связок $\vee$ и $\&$.

Для установления полного изоморфизма между ИИВ и просто типизированным $\lambda$-исчислением введём три необходимые для установления этого изоморфизма сущности:

\begin{enumerate}
	\item Тип "Ложь" ($\bot$)
	
	\item Тип упорядоченной пары $A\&B$, соответствующий	логическому "И"
	
	\item Алгебраический тип $A | B$, соответствующий логическому "ИЛИ"
\end{enumerate}

\subparagraph{Тип $\bot$}

Введём тип $\bot$, соответствующий лжи в ИИВ. Поскольку из лжи может следовать что угодно, добавим в исчисление новое правило вывода

\[
\infer{\Gamma \vdash A : \tau}{\Gamma \vdash A: \bot}
\]

То есть выражение, типизированное как $\bot$, может быть типизировано также любым другим типом.

В программировании аналогом этого типа может являться тип \mintinline{scala}{Nothing}, который является подтипом любого другого типа.

Тип \mintinline{scala}{Nothing} является необитаемым, им типизируется выражение, никогда не возвращающее свой результат (например, \mintinline{scala}{throw new Error() : Nothing}). 

Тот факт, что выражение, типизированное как \mintinline{scala}{Nothing}, может быть типизировано любым другим типом, позволяет писать следующие функции:

\begin{minted}{scala}
def assertStringNotEmpty(s: String): String = {
  if (s.length != 0) {
    s
  } else {
    throw new Error("Empty string")
  }
}
\end{minted}

так как \mintinline{scala}{throw new Error("Empty string"): Nothing}, то 

\mintinline{scala}{throw new Error("Empty string"): String}, поэтому функция может иметь тип \mintinline{scala}{String}.

Теперь, имея тип $\bot$, можно ввести связку "Отрицание". Обозначим $\neg A = A \rightarrow \bot$, то есть в программировании это будет соответствовать функции

\begin{minted}{scala}
def throwError(a: A): Nothing = throw new Error()
\end{minted}

\subparagraph{Упорядоченные пары}

Введём возможность запаковывать значения в пары.

Функция $makePair$ будет выглядеть следующим образом:

$$makePair \equiv \lambda \; first. \; \lambda \; second. \; \lambda \; f. \; f \; first \; second$$\newline

Тогда 

$$<first, second> \equiv makePair \; first \; second$$

Надо также написать функции, которые будут доставать из пары упакованные в неё значения. Назовём их $\Pi_1$ и $\Pi_2$. 

Пусть 
$$\Pi_1 \equiv \lambda \; Pair. \; Pair \; (\lambda \; a. \lambda \; b. \; a)$$

$$\Pi_2 \equiv \lambda \; Pair. \; Pair \; (\lambda \; a. \lambda \; b. \; b)$$

Заметим, что 

\begin{align*}
	\Pi_1 <A, B> \\
	=_{\beta} (\lambda \; Pair. \; Pair \; (\lambda \; a. \lambda \; b. \; a)) (makePair \; A \; B)\\ =_\beta (\lambda \; Pair. \; Pair \; (\lambda \; a. \lambda \; b. \; a)) (\textcolor{magenta}{(\lambda \; first. \; \lambda \; second. \; \lambda \; f. \; f \; first \; second)} \; \textcolor{blue}{A} \; B) \\ =_\beta (\lambda \; Pair. \; Pair \; (\lambda \; a. \lambda \; b. \; a)) (\textcolor{magenta}{(\lambda \; second. \; \lambda \; f. \; f \; A \; second)} \; \textcolor{blue}{B}) \\ =_\beta \textcolor{magenta}{(\lambda \; Pair. \; Pair \; (\lambda \; a. \lambda \; b. \; a))} \textcolor{blue}{(\lambda \; f. \; f \; A \; B)} \\ =_\beta \textcolor{magenta}{(\lambda \; f. \; f \; A \; B)} \; \textcolor{blue}{(\lambda \; a. \lambda \; b. \; a)} \\ =_\beta \textcolor{magenta}{(\lambda \; a. \lambda \; b. \; a)} \; \textcolor{blue}{A} \; B \\ =_\beta \textcolor{magenta}{(\lambda \; b. \; A)} \; \textcolor{blue}{B} \\ =_\beta A
\end{align*}

Аналогично, $\Pi_2 <A, B> =_\beta B$

Таким образом, мы умеем запаковывать элементы в пары и доставать элементы из пар. Теперь, добавим к просто типизированному $\lambda$-исчислению правила вывода, позволяющие типизировать такие конструкции.

Добавим три новых правила вывода:

\begin{enumerate}
	\item Правило типизации пары
	\[
	\infer{\Gamma \vdash <A, B>: \varphi \text{\&} \psi}{\Gamma\vdash A: \varphi && \Gamma\vdash B: \psi}
	\]
	\item Правило типизации первого проектора:
	\[
	\infer{\Gamma \vdash \Pi_1 <A, B> : \varphi}{\Gamma \vdash <A, B>: \varphi \text{\&} \psi}
	\]
	
	\item Правило типизации второго проектора:
	\[
	\infer{\Gamma \vdash \Pi_2 <A, B> : \psi}{\Gamma \vdash <A, B>: \varphi \text{\&} \psi}
	\]
\end{enumerate}

\subparagraph{Алгебраические типы}

Добавим тип, который является аналогом \mintinline{C++}{union} в C++, или алгебраического типа в любом функциональном языке. Это тип, который может содержать одну из двух альтернатив.

Например, тип \mintinline{scala}{OptionInt = None | Some of Int} может содержать либо \mintinline{scala}{None}, либо \mintinline{scala}{Some of Int}, но не обе альтернативы разом, причём в каждый момент времени известно, какую альтернативу он содержит.

Заметим, что определение алгебраического типа похоже на определение дизъюнкции в ИИВ (в ИИВ если выполнено $\vdash a \vee b$, известно, что из $\vdash a$ и $\vdash b$ выполнено).

Для реализации алгебраических типов в $\lambda$-исчислении напишем три функции:

\begin{enumerate}
	\item $in_1$, создающее экземпляр алгебраического типа из первой альтернативы, то есть запаковывающее первую альтернативу в алгебраический тип
	
	\item $in_2$, выполняющее аналогичные действия, но со второй альтернативой.
	
	\item $case$, принимающую три параметра: экземпляр алгебраического типа, функцию, определяющую, что делать, если этот экземпляр был создан из первой альтернативы (то есть с использованием $in_1$), и функцию, определяющую, что делать, если этот экземпляр был создан из второй альтернативы (то есть с использованием $in_2$)
\end{enumerate}

Аналогом $case$ в программировании является конструкция, известная как pattern-matching, или сопоставление с образцом.

\begin{minted}{ocaml}
let isEmptyList list = match list with
	| Nil -> true
	| Cons(_, _) -> false
;;
\end{minted}

Функция $in_1$ будет выглядеть следующим образом: 

$$in_1 \equiv \lambda \; x. \; \lambda \; f. \; \lambda \; g. \; f \; x$$

А $in_2$ -  следующим:

$$in_2 \equiv \lambda \; x. \; \lambda \; f. \; \lambda \; g. \; g \; x$$

То есть $in_1$ принимает две функции и применяет первую к $x$, а $in_2$ применяет вторую.

Тогда $case$ будет выглядеть следующим образом:

$$case \equiv \lambda \; algebraic. \; \lambda \; f. \; \lambda \; g. \; algebraic \; f \; g$$

Заметим, что 

\begin{align*}
case \; (in_1 A) \; F \; G \\ =_\beta (\lambda \; algebraic. \; \lambda \; f. \; \lambda \; g. \; algebraic \; f \; g) \; (\textcolor{magenta}{(\lambda \; x. \; \lambda \; h. \; \lambda \; s. \; h \; x)} \textcolor{blue}{A}) \; F \; G \\ =_\beta  \textcolor{magenta}{(\lambda \; algebraic. \; \lambda \; f. \; \lambda \; g. \; algebraic \; f \; g)} \; \textcolor{blue}{(\lambda \; h. \; \lambda \; s. \; h \; A)} \; F \; G \\ =_\beta \textcolor{magenta}{(\lambda \; f. \; \lambda \; g. \; (\lambda \; h. \; \lambda \; s. \; h \; A) \; f \; g)} \; \textcolor{blue}{F} \; G \\ =_\beta \textcolor{magenta}{(\lambda \; g. \; (\lambda \; h. \; \lambda \; s. \; h \; A) \; F \; g)} \; \textcolor{blue}{G}  \\ =_\beta \textcolor{magenta}{(\lambda \; h. \; \lambda \; s. \; h \; A)} \; \textcolor{blue}{F} \; G \\ =_\beta \textcolor{magenta}{(\lambda \; s. \; F \; A)} \; \textcolor{blue}{G} \\ =_\beta F \; A
\end{align*}

Аналогично, $case \; (in_2 B) \; F \; G =_\beta G \; B$. 

То есть $case, in_1$ и $in_2$ умеют применять нужную функцию к запакованной в экземпляр алгебраического типа одной из альтернатив.

Теперь добавим к просто типизированному $\lambda$-исчислению правила вывода, позволяющие типизировать эти конструкции.

Добавим три новых правила вывода:

\begin{enumerate}
	\item Правило типизации левой инъекции
	\[
	\infer{\Gamma \vdash in_1 \; A: \varphi \vee \psi}{\Gamma\vdash A: \varphi}
	\]
	\item Правило типизации правой инъекции:
	\[
	\infer{\Gamma \vdash in_2 \; B: \varphi \vee \psi}{\Gamma\vdash B: \psi}
	\]
	
	\item Правило типизации case:
	\[
	\infer{case \; L \; f \; g : \tau}{\Gamma \vdash L: \varphi \vee \psi, \;\;\;\; \Gamma \vdash f : \varphi \to \tau, \;\;\;\; \Gamma \vdash g: \psi \to \tau}
	\]
\end{enumerate}

\subsection{Изоморфизм Карри-Ховарда для расширения \\просто типизированного $\lambda$-исчисления}

Заметим точное соответствие только что введённых конструкций аксиомам ИИВ.

$\newline$
\begin{tabular}{ | p{8.5cm} | p{7.5cm} | }
	\hline
	Расширенное просто типизированное &ИИВ\\$\lambda$-исчисление &\\ \hline
	$\infer{\Gamma \vdash <A, B>: \varphi \text{\&} \psi}{\Gamma\vdash A: \varphi && \Gamma\vdash B: \psi}$ & $\vdash \varphi \to \psi \to \varphi \text{\&} \psi$ \\
	&\\
	$\infer{\Gamma \vdash \Pi_1 <A, B> : \varphi}{\Gamma \vdash <A, B>: \varphi \text{\&} \psi}$ & $\vdash \varphi \text{\&} \psi \to \varphi$  \\
	&\\
	$\infer{\Gamma \vdash \Pi_2 <A, B> : \psi}{\Gamma \vdash <A, B>: \varphi \text{\&} \psi}$ & $\vdash \varphi \text{\&} \psi \to \psi$  \\	
	&\\	
	$\infer{\Gamma \vdash in_1 \; A: \varphi \vee \psi}{\Gamma\vdash A: \varphi}$ & $\vdash \varphi \to \varphi \vee \psi$ \\
	&\\
	$\infer{\Gamma \vdash in_2 \; B: \varphi \vee \psi}{\Gamma\vdash B: \psi}$ & $\vdash \psi \to \varphi \vee \psi$ \\
	&\\
	$\infer{case \; L \; f \; g : \tau}{\Gamma \vdash L: \varphi \vee \psi, \;\; \Gamma \vdash f : \varphi \to \tau, \;\; \Gamma \vdash g: \psi \to \tau}$ & $\vdash  (\varphi \to \tau) \to (\psi \to \tau) \to (\varphi \vee \psi) \to \tau$ \\
	\hline
\end{tabular}

$\newline$

\subsection{Просто типизированное по Чёрчу $\lambda$-исчисление}

\begin{definition}
	Тип в просто типизированном по Чёрчу $\lambda$-исчислении --- это то же самое, что тип в просто типизированном по Карри $\lambda$-исчислении
\end{definition}

\begin{definition}
	Язык просто типизированного по Чёрчу $\lambda$-исчисления удовлетворяет следующей грамматике
	
	$\Lambda_{\text{ч}} ::= x \; | \; \Lambda_{\text{ч}} \; \Lambda_{\text{ч}} \; | \; \lambda\; x^\tau. \; \Lambda_{\text{ч}} \; | \; (\Lambda_{\text{ч}})$
\end{definition}

\begin{note}
	Иногда абстракция записывается не как $\lambda\; x^\tau. \; \Lambda_{\text{ч}}$, а как $\lambda\; x : \tau. \; \Lambda_{\text{ч}}$
\end{note}

\begin{definition}
	Просто типизированное по Чёрчу $\lambda$-исчисление.
	
	Рассмотрим исчисление с единственной схемой аксиом:
	
	$$\Gamma, x : \theta \vdash x : \theta, \text{если } x \text{ не входит в } \Gamma$$
	
	И следующими правилами вывода
	
	\begin{enumerate}
		\item Правило типизации абстракции
		\[
		\infer[\text{если } x \text{ не входит в } \Gamma]{\Gamma \vdash (\lambda \; x : \varphi. \; P) : \varphi \rightarrow \psi}{\Gamma, x : \varphi \vdash P : \psi}
		\]
		\item Правило типизации аппликации:
		\[
		\infer{\Gamma \vdash PQ : \psi}{\Gamma \vdash P : \varphi \to \psi && \Gamma \vdash Q : \varphi}
		\]
	\end{enumerate}
	
	Если $\lambda$-выражение типизируется с использованием этих двух правил и одной схемы аксиом, то будем говорить, что оно типизируется по Чёрчу.
\end{definition}

В исчислении по Чёрчу остаются верными все предыдущие теоремы (в том числе теорема Чёрча-Россера), но правило строгой типизации абстракций позволяет доказать ещё одну теорему:

\begin{theorem}[Уникальность типов в исчислении по Чёрчу] \ 
	\begin{enumerate}
		\item Если $\Gamma \vdash_{\text{ч}} M : \theta$ и $\Gamma \vdash_{\text{ч}} M : \tau$, то $\theta = \tau$
		\item Если $\Gamma \vdash_{\text{ч}} M : \theta$ и $\Gamma \vdash_{\text{ч}} N : \tau$, и $M =_\beta N$ то $\theta = \tau$
	\end{enumerate}
\end{theorem}

\subsection{Связь типизации по Чёрчу и по Карри}

\begin{definition}[Стирание] Функцией стирания называется следующая функция: 
	
$|\cdot| : \Lambda_{\text{ч}} \to \Lambda_{\text{к}}$:
	\[
	|{A}| =
	\begin{cases}
	x                                   & A \equiv x \\
	|M| \; |N| & A \equiv M \; N \\
	\lambda x . |P|           & A \equiv \lambda \; x : \tau. \; P
	\end{cases}
	\]
\end{definition}

\begin{lemma}
	Пусть $M, N \in \Lambda_{\text{ч}}, M \to_\beta N$, тогда $|M| \to_\beta |N|$
\end{lemma}

\begin{lemma}
	Если $\Gamma \vdash_{\text{ч}} M : \tau$, тогда $\Gamma' \vdash_{\text{к}} |M| : \tau$, где $\Gamma'$ получается из $\Gamma$ применением функции стирания к каждому терму из $\Gamma$
\end{lemma}

\begin{theorem}[Теорема о поднятии] \ 
	\begin{enumerate}
		\item Пусть $M, N \in \Lambda_{\text{к}}, P \in \Lambda_{\text{ч}}, |P| = M, M \to_\beta N$. Тогда найдётся такое $Q \in \Lambda_{\text{ч}}$, что $|Q| = N$, и $P \to_\beta Q$
		\item Пусть $M \in \Lambda_{\text{к}}, \Gamma \vdash_{\text{к}} M : \tau$. Тогда существует $P \in \Lambda_{\text{ч}}$, что $|P| = M$, и $\Gamma \vdash_{\text{ч}} P : \tau$
	\end{enumerate}
\end{theorem}
